% In this file you should put all LaTeX macros and settings to be used both by
% the pdf version and the web version.
% This should be most of your macros.

% The theorem-like environments defined below are those that appear by default
% in the dependency graph. See the README of leanblueprint if you need help to
% customize this. 
% The configuration below use the theorem counter for all those environments
% (this is what the [theorem] arguments mean) and never resets it.
% If you want for instance to number them within chapters then you can add
% [chapter] at the end of the next line.

% \newtheorem{theorem}{Theorem}
% \newtheorem{proposition}[theorem]{Proposition}
% \newtheorem{lemma}[theorem]{Lemma}
% \newtheorem{corollary}[theorem]{Corollary}
% \newtheorem{definition}[theorem]{Definition}


\newcommand{\Z}{\mathbb{Z}}
\newcommand{\N}{\mathbb{N}}
\newcommand{\A}{\mathbb{A}}
\newcommand{\Q}{\mathbb{Q}}
\newcommand{\R}{\mathbb{R}}
\newcommand{\F}{\mathbb{F}}
\newcommand{\Fbar}{\bar{\F}}
% \newcommand{\Qp}{\mathbb{Q}_p}
% \newcommand{\Ql}{\mathbb{Q}_\ell}
% \newcommand{\Qbar}{\overline{\Q}}
% \newcommand{\Qpbar}{\overline{\Q}_p}
% \newcommand{\Qlbar}{\overline{\Q}_\ell}
\newcommand{\C}{\mathbb{C}}
% \newcommand{\bigslant}[2]{{\raisebox{.2em}{$#1$}\left/\raisebox{-.2em}}}
% \newcommand{\GQ}{\Gal(\Qbar/\Q)}
% \newcommand{\GQp}{\Gal(\Qpbar/\Qp)}
% \newcommand{\GQl}{\Gal(\Qlbar/\Ql)}
% \newcommand{\m}{\mathfrak{m}}
% \newcommand{\GK}{\Gal(K^{\sep}/K)}
% \newcommand{\GN}{\Gal(\overline{N}/N)}
% \newcommand{\Kbar}{\overline{K}}
% \newcommand{\Qhat}{\widehat{\Q}}
% \newcommand{\calO}{\mathcal{O}}
% \newcommand{\calOhat}{\widehat{\calO}}
% \newcommand{\bbH}{\mathbb{H}}
% \newcommand{\p}{{\mathfrak{p}}}
% \newcommand{\rhobar}{\overline{\rho}}
% \newcommand{\Zhat}{\widehat{\Z}}
\newcommand{\inv}{^{\raisebox{.2ex}{$\scriptscriptstyle-1$}}}
\DeclareMathOperator{\Gal}{Gal}
\DeclareMathOperator{\avoid}{avoid}
\DeclareMathOperator{\Aut}{Aut}
\DeclareMathOperator{\GL}{GL}
\DeclareMathOperator{\PGL}{PGL}
\DeclareMathOperator{\PSL}{PSL}
\DeclareMathOperator{\SL}{SL}
\DeclareMathOperator{\Spec}{Spec}
\DeclareMathOperator{\sep}{sep}
\DeclareMathOperator{\ab}{ab}
\DeclareMathOperator{\tr}{tr}
\DeclareMathOperator{\Hom}{Hom}
\DeclareMathOperator{\Frob}{Frob}