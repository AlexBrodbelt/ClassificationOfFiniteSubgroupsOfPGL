\documentclass[a4paper , 11pt]{book}

\usepackage{amsmath}
\usepackage[english]{babel}
\usepackage{xcolor}
\usepackage{amsfonts}
\usepackage{amssymb}
\usepackage{amsthm}
\usepackage[utf8]{inputenc}
\usepackage[colorinlistoftodos]{todonotes}
\usepackage{graphicx}
\usepackage{mathrsfs}
\usepackage{amsbsy}
\usepackage{array}
\usepackage{tabu}
\usepackage[nottoc]{tocbibind}
\usepackage[toc,page]{appendix}
\usepackage{lipsum,appendix}
\usepackage{tikz}
\usepackage[margin=1.5in]{geometry}

\newcommand{\minus}{\scalebox{0.75}[1.0]{$-$}}

\newtheorem{theorem}{Theorem}[chapter]
\newtheorem*{theorem-non}{Theorem}

\newtheorem{corollary}[theorem]{Corollary}
\newtheorem{lemma}[theorem]{Lemma}
\newtheorem{proposition}[theorem]{Proposition}

\theoremstyle{definition}
\newtheorem{definition}{Definition}[section]
\newtheorem*{definition}{Definition}

\theoremstyle{remark}
\newtheorem*{remark}{Remark}

\newtheorem{notation}{Notation}
\newtheorem*{notation-non}{Notation}

\newtheorem*{lagrange}{\emph{\textbf{Lagrange's Theorem}}}
\newtheorem*{cauchy}{\emph{\textbf{Cauchy's Theorem}}}
\newtheorem*{1stiso}{\emph{\textbf{First Isomorphism Theorem}}}
\newtheorem*{2ndiso}{\emph{\textbf{Second Isomorphism Theorem}}}
\newtheorem*{3rdiso}{\emph{\textbf{Third Isomorphism Theorem}}}
\newtheorem*{1stsylow}{\emph{\textbf{First Sylow Theorem}}}
\newtheorem*{2ndsylow}{\emph{\textbf{Second Sylow Theorem}}}
\newtheorem*{3rdsylow}{\emph{\textbf{Third Sylow Theorem}}}
\newtheorem*{4thsylow}{\emph{\textbf{Fourth Sylow Theorem}}}
\newtheorem*{orbstab}{\emph{\textbf{Orbit-Stabiliser Theorem}}}

\newcolumntype{P}[1]{>{\centering\arraybackslash}p{#1}}

\setlength{\parindent}{0pt}

\newcommand{\inv}{^{\raisebox{.2ex}{$\scriptscriptstyle-1$}}}

\begin{document}

\frontmatter
\title{Dickson's Classification of Finite Subgroups of the Two-dimensional Special Linear Group over an Algebraically Closed Field}
\author{Christopher Butler}
\date{\today}
\maketitle

\cleardoublepage


\tableofcontents

\newpage\phantom{blabla}
\thispagestyle{plain}
\mainmatter

\setcounter{chapter}{-1}

\newpage
\thispagestyle{plain}
\begin{center}
       \Large \textbf{Introduction}
\end{center}
\addcontentsline{toc}{section}{Introduction}

The general linear group of degree $n$ is the group formed by the set of $n$ x $n$ invertible matrices, together with the operation of ordinary matrix multiplication, with the entries of each matrix coming from a specific ring or field. The special linear group is a subgroup of the general linear group, namely those matrices with a deteminant equal to 1. In this work, we focus on the two-dimensional case, with entries coming from an algebraically closed field, $F$. This is denoted by $GL(2,F)$ for the general linear group and $SL(2,F)$ for the special linear group. Recall that an algebraically closed field is a field which contains the roots to any non-constant polynomial in $F[x]$, with coefficients in $F$. They are infinite in size and two such examples are the field of complex numbers and the field of algebraic numbers. \\
\vspace{-0.2mm}
\\
In 1901, Leonard Eugene Dickson published his book \textit{Linear Groups, with an Exposition of the Galois Field Theory} \cite{dickson}. In this work, he obtains a complete classification of the finite subgroups of $SL(2,F)$. This paper is a reformulation of Dickson's classification theorem and loosely follows Chapter 3, $\mathsection$6 in Michio Suzuki’s book \textit{Group Theory I} \cite{suzuki}. This classification theorem is of particular interest in the study of finite simple groups and Suzuki himself describes it as \textit{one of the indispensable tools in studying the basic properties of linear groups which underlie the concept of $p$-stability} \cite[p.392]{suzuki}. \\
\vspace{-0.2mm}
\\
The paper begins with a brief overview of some preliminary requirements which are necessary to the understanding of the proof. They are standard group theory results which may or may not have been covered in a first course given on group theory, the majority of which are cited without proof. A more advanced reader may choose to skip over this chapter. \\
\vspace{-0.2mm}
\\
The main body of work begins in Chapter 1 and focuses on the infinite group $SL(2,F)$. We make some important observations about the conjugacy of the elements in this group and the centre of the group. Some important elements and subgroups of $SL(2,F)$ are defined and their centralisers and normalisers determined. We show that the action of $SL(2,F)$ on the projective line is triply transitive, which is a vital tool used several times throughout the paper in determining group structure. \\
\vspace{-0.2mm}
\\
In Chapter 2 we consider an arbitrary finite subgroup $G$ of $SL(2,F)$. The notion of a \textit{maximal abelian subgroup} is introduced and utilised to construct a class equation, whereby $G$ is partitioned into the conjugacy classes of it's maximal abelian subgroups. This plays a crucial role in determining the possible structures of $G$. We find that the number and type of these conjugacy classes are restricted to just 6 different cases. \\
\vspace{-0.2mm}
\\
The final chapter examines these 6 cases individually. In each case we determine the possible structures that $G$ could have. The 10 possible structures of $G$ are finally consolidated into the classification theorem.


\newpage\phantom{blabla}
\thispagestyle{plain}

\chapter[Preliminaries]{Preliminaries}
\chaptermark{Preliminaries}

% done

\chapter[Properties of $\pmb{SL(2,F)}$ over an Algebraically Closed Field]{Properties of $\pmb{SL(2,F)}$ over an Algebraically Closed Field}
\chaptermark{Algebraically Closed Field}


% done






\begin{thebibliography}{9}

\bibitem{alperin} 
Alperin, J.L., Bell, R.B. 
\textit{Groups and Representations}. 
Springer,
(1995).

\bibitem{bhattacharya} 
Bhattacharya, P.B., Jain, S.K., Nagpaul, S.R. 
\textit{Basic Abstract Algebra, Second Edition}. 
Cambridge University Press,
(1994).

\bibitem{dickson} 
Dickson, L.E. 
\textit{Linear Groups, with an Exposition of the Galois Field Theory}. 
B.G.Teubner, Leipzig,
(1901).

\bibitem{dummit} 
Dummit, D.S., Foote, R.M. 
\textit{Abstract Algebra}. 
Wiley,
(2004).

\bibitem{matrix} 
Holst, A., Ufnarovski, V. 
\textit{Matrix Theory}. 
Studentlitteratur,
(2014).

\bibitem{hungerford} 
Hungerford, T.W. 
\textit{Abstract Algebra: An Introduction, Third Edition}. 
Brooks/Cole, Cengage Learning,
(2014).

\bibitem{schur} 
Schur, I. 
\textit{Über die Darstellung der symmetrischen und der alternierenden Gruppe durch gebrochene lineare Substitutionen.} Journal für die reine und angewandte Mathematik (Crelles Journal) (139), p.155-250. 
De Gruyter,
(1911).

\bibitem{stewart} 
Stewart, I. 
\textit{Galois Theory, Third Edition}. 
Chapman \& Hall/CRC,
(2003).

\bibitem{suzuki}
Suzuki, M. 
\textit{Group Theory I}. 
Spinger-Verlag, Berlin, Heidelberg, New York, 
(1982).

\end{thebibliography}
\end{document}
