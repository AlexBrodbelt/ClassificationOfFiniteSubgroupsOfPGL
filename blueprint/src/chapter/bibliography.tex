
\chapter{Bibliography}

% \bibliographystyle{plain} % We choose the "plain" reference style
% \bibliography{bibliography}

\begin{thebibliography}{3}

    \bibitem{butler}
    Butler, C. (2019). 
    \textit{Dickson's Classification of Finite Subgroups of the Two-dimensional Special Linear Group over an Algebraically Closed Field.}
    Master's Theses in Mathematical Sciences 2019: E63.
    
    \bibitem{sangwin}
    Sangwin, C. (2023). 
    \textit{Sums of the first n odd integers.}
    The Mathematical Gazette, 107(568), 10-24.
    
    \bibitem{dtt}
    Henri Darmon, Fred Diamond, and Richard Taylor. 
    \textit{Fermat’s last theorem}. 
    In Current developments in mathematics, 
    1995 (Cambridge, MA), pages 1–154. Int. Press, Cambridge, MA, 1994.
    \bibitem{alperin} 
    Alperin, J.L., Bell, R.B. 
    \textit{Groups and Representations}. 
    Springer,
    (1995).
    
    \bibitem{bhattacharya} 
    Bhattacharya, P.B., Jain, S.K., Nagpaul, S.R. 
    \textit{Basic Abstract Algebra, Second Edition}. 
    Cambridge University Press,
    (1994).
    
    \bibitem{dickson} 
    Dickson, L.E. 
    \textit{Linear Groups, with an Exposition of the Galois Field Theory}. 
    B.G.Teubner, Leipzig,
    (1901).
    
    \bibitem{dummit} 
    Dummit, D.S., Foote, R.M. 
    \textit{Abstract Algebra}. 
    Wiley,
    (2004).
    
    \bibitem{matrix} 
    Holst, A., Ufnarovski, V. 
    \textit{Matrix Theory}. 
    Studentlitteratur,
    (2014).
    
    \bibitem{hungerford} 
    Hungerford, T.W. 
    \textit{Abstract Algebra: An Introduction, Third Edition}. 
    Brooks/Cole, Cengage Learning,
    (2014).
    
    \bibitem{schur} 
    Schur, I. 
    \textit{Über die Darstellung der symmetrischen und der alternierenden Gruppe durch gebrochene lineare Substitutionen.} Journal für die reine und angewandte Mathematik (Crelles Journal) (139), p.155-250. 
    De Gruyter,
    (1911).
    
    \bibitem{stewart} 
    Stewart, I. 
    \textit{Galois Theory, Third Edition}. 
    Chapman \& Hall/CRC,
    (2003).
    
    \bibitem{suzuki}
    Suzuki, M. 
    \textit{Group Theory I}. 
    Spinger-Verlag, Berlin, Heidelberg, New York, 
    (1982).
    
\end{thebibliography}