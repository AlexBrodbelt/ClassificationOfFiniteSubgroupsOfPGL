\chapter{Abstract and Acknowledgements}\label{Ch1_AbstractAndAcknowledgements}

\section{Acknowledgements}

I thank my supervisor Prof. David Jordan for his invaluable support and guidance throughout the project,

I would also like to thank Christopher Butler for providing the TeX code so I could easily set up the blueprint, and hopefully, improve and add to his amazing exposition of
\textbf{Dickson's Classification Theorem}.

I would also like to thank Prof. Kevin Buzzard for his support, patience and guidance throughout the project. His advice and comments on how I should go about formalising mathematics have been of utmost value. 

Finally, I would like to thank the many members of the Lean Zulip community who have provided insightful ideas and comments that have helped me progress much faster than otherwise, 
this also includes assistance with technical issues with setting up the blueprint and so forth. I am grateful to:

\begin{itemize}
    \item Artie Khovanov
    \item David Loeffler
    \item Mitchell Lee
    \item Yakov Pechersky
    \item Edward van de Meent
    \item Ruben Van de Velde
    \item Andrew Yang
    \item Johan Commelin
    \item Scott Carnahan
    \item Damiano Testa
    \item Aron Liu
\end{itemize}

\section{Abstract}

This project is a blueprint for the ongoing formalization of the classification of finite subgroups of $\PGL_2(\Fbar_p)$. 
The project brings together both informal, or pen-and-paper mathematics; alongside the formal mathematics into one cohesive interactive website that hopes to complement the strength of both
forms of mathematics.


\section{How to read this blueprint}



\section{Christopher Butler's acknowledgements and popular science summary}

Considering this project hinges very heavily on the work of Christopher Butler, I feel obliged to include his own acknowledgements, abstract and popular science summary on $\textbf{Dickson's Classification Theorem}$ for $\SL_2(F)$ over an algebraically closed field.

\subsection{Christopher Butler's Abstract }

This paper is a reformulation of Leonard Dickson's complete classification of the finite subgroups of the two-dimensional special linear group over an arbitrary algebraically closed field, $\SL_2(F)$. The approach is to construct a class equation of the conjugacy classes of maximal abelian subgroups of an arbitrary finite subgroup of $\SL_2(F)$. In turn, this leads to only 10 possible classes of structures of this subgroup up to isomorphism.

\subsection{Acknowledgements from Christopher Butler}

I would like to take this opportunity to thank my advisor Arne Meurman. This paper would not have been possible without the guidance and insight he gave during our weekly discussions.


\subsection{Christopher Butler's popular science summary}

In order to explain what this paper is about, it is necessary to first define a few of the mathematical concepts which it concerns. A \textit{group} is a set of objects, called \textit{elements}, together with a rule, called an \textit{operation}, which tells us how two elements combine with each other to make a third. Furthermore, to be considered a group it must also satisfy 4 conditions, called \textit{axioms}. One of which is that the group must be \textit{closed} under it's operation. This means that whenever any two elements in the group are combined, the resulting element is also part of the group. The remaining axioms require that the group must also be \textit{associative}, have an \textit{identity} element and each element must have an \textit{inverse}. The way in which the elements in a group act with each other is called the group's \textit{structure}. If 2 groups have the same number of elements and share the same structure, then they are regarded as being \textit{isomorphic} to each other, which essentially means that they equivalent. Many everyday things can be regarded as groups, such as the symmetries of geometrical objects, or the number systems we use. \\
\\
The set of 2 x 2 matrices whose \textit{determinant} is equal to 1, together with the operation of ordinary matrix multiplication, forms a group called the \textit{special linear group}. This is a group because the product of 2 matrices has a determinant equal to the product of the determinants of the 2 matrices, so since 1 x 1 = 1, this new element also belongs to the group, hence the axiom of being closed is satisfied. Furthermore, it is crucial that the entries in the matrices are taken from a specified \textit{ring} or \textit{field}. Rings and fields are, like groups, abstract mathematical objects, albeit they satisy even more axioms than groups do. Crucially, rings and fields have both an additive and a multiplicative identity. \\
\\
This paper focuses on $\SL_2(F)$, which is the two-dimensional special linear group whose entries are taken from an \textit{algebraically closed} field. Algebraically closed fields are infinite in size, which means that the resulting special linear group is also infinite. A \textit{subgroup} of a group is simply a group with the added requirement that each of it's elements must also belong to the original group. Thus a finite subgroup of $\SL_2(F)$ is any finite set of elements belonging to this infinite group $\SL_2(F)$, which satisfy the 4 axioms of being a group. \\
\\
This paper classifies all the possible structures which a finite subgroup of $\SL_2(F)$ could have. The result has implications within the study of finite \textit{simple} groups. This classification was first done by American mathematician Leonard Eugene Dickson in 1901. The purpose of this reformulation is to make it accessible to a wider audience by providing a more detailed explanation at the various stages of the proof.


