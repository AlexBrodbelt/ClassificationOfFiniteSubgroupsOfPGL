\chapter{Introduction}\label{Ch2_Introduction}

\section{What is the formalization of mathematics?}

Formalization of mathematics is the art of teaching a computer what a piece of mathematics means, that is, it is the process of carefully writing down a mathematical statement typically in first order logic or higher order logic and then scrutinously justifying each step of the proof to a computer program that checks the validity of every step of the reasoning. 

TODO:

- Brief explanation of type theory and curry-howard isomorphism.

- Example of formal proof and comparison with informal proof.


% \begin{minted}{lean}
% theorem add_comm (a b : Nat) : a + b = b + a :=
%   Nat.add_comm a b
% \end{minted}


\section{Fermat's Last Theorem}

TODO:

-History of Fermat's Equation

-Problem statements

-Developments in number theory that lead to the resolution of the conjecture.


\section{Formalizing Fermat's Last Theorem}

Following the sequence of success stories ranging from the Liquid Tensor Experiment to the formalization of the Polynomial Freiman-Rusza conjecture. 

Prof. Kevin Buzzard from Imperial College London has received a five-year grant that will allow him to lead the formalization of Fermat's Last Theorem. This grant kicked in in October of 2024. 

At the time of writing, since October of 2024, a digital blueprint has been set up to manage the project.

Alongside other infrastructure like the project dashboard, mathematicians around the world can claim tasks that are set by Prof. Kevin Buzzard and if in return a task is returned with a "sorry" free proof then one can claim the glory of having completed the task.

TODO:

- Current goal of the formalization

- Explain somewhat the modern approach and the highly sought after Modularity Lifting Theorem.

- My task: Classification of finite subgroups of $\PGL_2(\bar{\F}_p)$



\section{Classification of finite subgroups of the Projective General Linear group over the algebraically closed field}

TODO:

-Why are the finite subgroups of  $\PGL_2(\bar{\F}_p)$ relevant to number theory: i.e: Automorphic forms, Galois representations, etc.


The primary concern of this project is to formalise Theorem 2.47 of [DTT] which states:

\begin{enumerate}
    \item If $H$ is finite subgroup of $\PGL_2(\C)$ then $H$ is isomorphic to one of the following groups: the cyclic group $C_n$ of order $n$ ($n \in \Z_{>0}$), the dihedral group $D_{2n}$ of order $2n$ ($n \in \Z_{>1}$), $A_4$, $S_4$ or $A_5$.
\item If $H$ is a finite subgroup of $\PGL_2(\Fbar_p)$ then one of the following holds:
\begin{enumerate}
    \item $H$ is conjugate to a subgroup of the upper triangular matrices;
    \item $H$ is conjugate to $\PGL_2(\F_{\ell^r})$ and $\PSL_2(\F_{\ell^{r}})$ for some $r \in \Z_{>0}$;
    \item $H$ is isomorphic to $A_4$, $S_4$, $A_5$ or the dihedral group $D_{2r}$ of order $2r$ for some $r \in \Z_{>1}$ not divisible by $\ell$

\end{enumerate}
    Where $\ell$ is assumed to be an odd prime.
\end{enumerate}


By definition the Projective General Linear Group is:

\begin{equation}
    \PGL_n(F) = \GL_n(F) / (Z(\GL_n(F))) = \GL_n(F) / (F^\times I) 
\end{equation}

Similarly, the Projective Special Linear Group is:

\begin{equation}
    \PSL_n(F) = \SL_n(F) / (Z(\SL_n(F))) = \SL_n(F) / (\langle -I\rangle)
\end{equation}

Given we are working over an algebraically closed field $F$, it turns out that for any $n \in \N$, $\PGL_n(F)$ is isomorphic to $\PSL_n(F)$.

This isomorphism will be crucial as it will allow us to focus on classifying finite subgroups of $SL_2(F)$ to classify the finite subgroups of $\PGL_2(F)$.

The goal of the next chapter is to prove and formalize this result.