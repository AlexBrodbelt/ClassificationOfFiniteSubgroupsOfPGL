\chapter{Introduction}\label{Ch2_Introduction}

\section{What is the formalisation of mathematics?}

Formalisation of mathematics is the art of teaching a computer what a piece of mathematics means.

That is, it is the process of carefully writing down a mathematical statement, typically in first order logic or higher order logic, 
and then scrutinously justifying each step of the proof to a computer program that checks the validity of every step of the reasoning. 

Typically, one formalizes mathematics with the help of a proof assistant or what is also know as an interactive theorem prover, 
a piece of software which enables a human to write down mathematics and have the software verify the claims.

There exist many proof assistants, such examples are Lean, Isabelle, Coq, Metamath, and so forth.

For this project I have opted to use Lean due to its rapid growing mathematics library and its dependent type theory. I shall attempt to explain in more detail these last two reasons indirectly through
examples, but first I will comment on what Lean is.

\subsubsection{What is Lean?}

Lean is both a functional programming language and an interactive theorem prover (also known as a proof assistant) that is being developed at Microsoft research and AWS by Leonardo de Moura and his team.
It has been designed for both use in cutting-edge mathematics and the verification of software, which is often essential to safety critical systems such as medical or aviation software; where any error can have
catastrophic consequences on people's lives or expensive infrastructure.

Theorem provers like Lean harness the tight bond between proofs and programs. Often an algorithm, in fact serves as a proof for a mathematical statement.

For example, such is the case for the following theorem:

\begin{example}[Algorithm corresponds to a proof - Bézout's lemma]
    Let $R$ be a ring with a euclidean function $\nu : R \setminus \{0\} \rightarrow \Z_{\geq 0}$ which satisfies that for all $x, y \in R$ with $y \ne 0$, there exist $q, r \in \R$ such that $a  = qb + r$ where either $r = 0$ or $\nu(r) < \nu(b)$.

    For any $r,s \in R$ to find a unique linear combination which is the greatest common divisor of $r$ and $s$, that is, there exist coefficients $a, b \in R$ such that $ar + bs = \gcd(r, s)$.
\end{example}
\begin{proof}
    We construct $a$ and $b$ by the extended euclidean algorithm, we sequentially divide in the following fashion:
      \begin{align}
          r &= q_0 s + r_1\\
          b &= q_1 r_1 + r_2\\
          r_1 &= q_2 r_2 + r_3\\
          &\vdots\\
          r_{i-1} &= q_i r_i + r_{i + 1}&
      \end{align}
  
      by the definition of a euclidean domain, we have a strictly decreasing sequence $\nu(r_1) > \nu(r_2) > \ldots > \nu(r_k)$ that must eventually terminate in at most $\nu(r_1) + 1$ steps,
      and must have that $\nu(r_k) = 0$ for some $k \in \N$. It will then be that $r_{k -1} = \gcd(r, s)$, and by back substitution we can recover the values for the coefficients $a$ and $b$.
  \end{proof}

The proof for this theorem uses the extended euclidean algorithm and proves that when the algorithm terminates it produces the desired witnesses.
In \texttt{mathlib}, the extended euclidean algorithm is defined in the following way and is used to formalise Bézout's lemma


It is not always clear how a proof corresponds to a program, but this correspondence does exist nonetheless.

The correspondence is known as the \textbf{Curry-Howard correspondence}, where formulas correspond to \textit{types}, which correspond to the notion of a specification;
proofs for formulas correspond to constructing a term of the corresponding type and so forth. 

In fact, it turns out that for every logic, such as classical or intuitionistic logic, there corresponds a type system which express the valid rules for programs. 

For our purposes, we will not provide a deep overview of this fundamental correspondence, but rather we will illustrate the core principle with a suitable example.

Before we delve in to another example, observe that within the block of Lean code above there was a lot of unfamiliar syntax which one is somehow meant to believe correspond to mathematics. 

The following example hopes to illustrate a simpler example and give a very brief overview of how to:

\begin{itemize}
    \item Define the assumptions for a mathematical statement.
    \item Define the mathematical statement.
    \item Formalise the mathematical statement using Lean tactics.
\end{itemize}

Note that in the list above, the notion of a Lean tactic is mentioned. Loosely speaking, a \texttt{tactic} is a Lean metaprogram that will write Lean programs, these (non-meta) programs can
be the usual code one would write for an algorithm or it could be the program which corresponds to the proof term that one needs to construct to formalise
a mathematical statement in Lean. Examples of tactics and how they are used will be illustrated in the following example, yet a comprehensive list of Lean tactics can be found at \href{https://lean-lang.org/doc/reference/latest//Tactic-Proofs/Tactic-Reference/}{Tactics}.

\begin{example}[Proving and formalising the sum of the first $n$ odd integers]
    To understand how the nature of proof is preserved when passed into a theorem prover, we will compare side by side the informal and formal proofs 
    for why the sum of the first $n$ odd integers equals the $n$th square. That is, we will prove and formalise:

    \begin{equation}
        \label{ih}
        \sum_{k = 1}^{n} 2k - 1 = n^2
    \end{equation}

    There are many ways to prove this statement, other proofs can be found at \cite{sangwin}. The proof that is best suited to be formalised is the
    proof by induction which goes as the following:

    \begin{proof}
        We prove the claim holds for all $n \in \N$ by the principle of mathematical induction. Indeed,

        \begin{itemize}
            \item The claim holds true for $n = 1$ since the LHS is $\sum_{k = 1}^{1} 2k -1 = 1$ and the RHS is $1^2 = 1$ and indeed $\textrm{LHS} = \textrm{RHS}$. This proves the base case.
            
            \item Let $m \in \N$ be a natural number and suppose the statement \eqref{ih} holds for $n = m$ then we will show that it then follows that it must hold for $n = m + 1$. Indeed,
            
            Consider the sum $\sum_{k = 1}^{m + 1} 2k - 1$, then we have that
            \begin{align}
                \sum_{k = 1}^{m + 1} 2k - 1 &= \left(\sum_{k = 1}^{m} 2k - 1\right) + 2(m + 1) - 1 \tag{by definition of the summation}\\
                    &=  m^2 + 2m + 1 \tag{by the induction hypothesis}\\
                    &= (m + 1)^2
            \end{align}

            This proves the induction step, and therefore by the principle of mathematical induction. The claim holds true for all $n \in \N$.
        \end{itemize}
    \end{proof}

    To define this statement in Lean we first must define what we mean by $\sum_{k = 1}^{n} 2k - 1$, to define this sum in Lean we use the recursive definition
    for the summation where 

    \begin{align*}
        \sum_{k = 1}^{n + 1} f(k) &= \sum_{k = 1}^{n} f(k) + f(n + 1) \tag{for $n \geq 0$}\\
            &\text{and}\\
        \sum_{k = 1}^{0} f(k) &= 0
    \end{align*}

    Where it is noteworthy to add that in Lean that naturals numbers include zero.

    This definition above which sums the odd numbers is equivalent up to re-indexing to the definition above. 
    
    The reason we do not use subtraction is because the natural numbers are a commutative semiring, so 
    it does not always make sense to subtract a natural number from a another natural number. For instance, the predecessor of
    zero is not defined. 
    
    Instead to understand a generic natural number we only need understand that a natural number is either zero or a successor of a natural number.
    Thus, when defining a function from the naturals,
    we only need to think about where to send zero and where to send the successor of a natural number, such a definition for a function
    are often denoted as inductive/recursive definitions. This pattern of thought is continually used throughout Lean's inductive types.

    Given the code definition above is a program one can indeed compute using the
    function defined in this programming language!

    In fact, this might be how one would first conjecture that such a theorem about 
    the sums of the first odd natural numbers might be true in the first place!

    To state the theorem in Lean, we use the keyword \texttt{theorem} or \texttt{lemma};
    followed by the name we would like to give the theorem, in this case it is, \texttt{closed\_eq\_sum\_of\_first\_n\_odd\_nat};
    then followed by a list of arguments which will either be the objects or assumptions on the objects, in this case we only
    specify that \texttt{n} is a natural number; then after a colon \texttt{:}, we specify the mathematical statement,
    in this case that \texttt{sum\_of\_first\_n\_odd\_nat n = n * n}. We have stated our first mathematical statement in Lean!

    Now we will walk through the formal proof in Lean:

    After the \texttt{:=} Lean expects a proof term of the type \texttt{sum\_of\_first\_n\_odd\_nat n = n * n}. It is possible to
    define the corresponding program which constructs the term, but often it is more intuitive to enter what is known as \textit{tactic mode}.

    As outlined above, tactics are metaprograms, programs that write programs, which in this case allow the simulation of typical pen-and-paper mathematics
    in Lean; to enter tactic mode one must begin the proof with the \texttt{by} tactic. A benefit of using tactic mode is that it allows access to an extremely useful
    interactive \textit{infoview} which displays the objects at play in the proof, the assumptions on the objects and the state of the proof.

    Once in tactic mode, we have access to other tactics accessible through the keywords like: \texttt{induction}, \texttt{case}, \texttt{rw}, \texttt{rw}, \texttt{ring}, \texttt{simp} and so on.
    
    Given the natural numbers are defined inductively in Lean, Lean understands that to prove that a property $P$ holds true for all natural numbers
    it is sufficient to provide a proof term for $P(0)$ and supposing $P(n)$ holds we can show that then $P(n+1)$ holds, in Lean terminology, the natural numbers
    have their own induction principle; notice the similarity to defining a function from the naturals. In fact, the generation of an induction principle 
    will be automatically true for any inductive datatype, but we will not go into this.
    
    To access this fact, we must invoke the \texttt{induction} tactic which splits the original goal of \texttt{sum\_of\_first\_n\_odd\_nat n = n * n} into two smaller goals

    \begin{enumerate}
        \item The base case: \texttt{sum\_of\_first\_n\_odd\_nat 0 = 0 * 0}.
        \item The induction step: \texttt{sum\_of\_first\_n\_odd\_nat (m + 1) = (m + 1) * (m + 1)}
    \end{enumerate}

    The \texttt{case} tactic allows us to focus in on one of the tactics, at first we focus on the goal with the label \textit{zero} to prove the base case;
    then we focus in on the induction step by typing \texttt{case succ m hm} which also introduces two new objects into the proof context, the natural number $m$ and the assumption
    on \texttt{m} which says that $m$ satisfies the induction hypothesis, \texttt{sum\_of\_first\_n\_odd\_nat m = m * m}. 
    
    We then proceed to use the rewrite tactic, \texttt{rewrite}, which allows to replace equal or logically equivalent terms, so if you have the theorem \texttt{h : a = b}, then \texttt{rw [h]} will replace every occurrence of \texttt{a} in the goal
    for a \texttt{b}, in the new modified goal. For this particular example, the proof for the base case labelled by \texttt{zero} involves using the rewrites \texttt{mul_zero} which states that \texttt{a * 0 = 0} and
    the definition of  \texttt{sum\_of\_first\_n\_odd\_nat} itself, where Lean is clever enough to choose which of two equalities to rewrite.

    Finally, \texttt{rfl} proves any goal that is true by reflexivity of the given relation; in this case, we finish proving the goal by reflexivity of the equality relation. Typically, one
    uses the \texttt{rw} tactic which is a combination of \texttt{rewrite} followed by \texttt{rfl}.

    Theorems and lemmas in Lean are given an identifier by which to access through, for example, the theorem \texttt{add\_mul\_self\_eq} states that for all $a, b \in S$ where $S$ is a semiring
    we have that \texttt{(a + b) * (a + b) = a * a + 2 * a * b + b * b}.
\end{example}


\section{Fermat's Last Theorem}


\subsubsection{Problem statement and its history}
Fermat's Last Theorem, before it was proved that is, A conjecture about the \textit{Fermat equation} which is defined to be

\begin{definition}[Fermat Equation]
    The equation $a^n + b^n = c^n$ is Fermat's Equation
\end{definition}

When $a, b, c$ and $n$ in this equation are restricted to positive integers, we are defining a particular family of what are called \textit{Diophantine equation}.
Diophantus, an ancient greek mathematician was interested in positive integers which satisfy this equation. For instance, a particular set of numbers which satisfy this equation 
are the \textit{Pythagorean triples}, such triples have been known since Babylonian times. For example, when we substitute the Pythagorean triple $(a,b,c) = (3,4,5)$ and set $n = 2$ we find that 
indeed Fermat's equation holds for this choice of numbers since:

\[
3^2 + 4^2 = 5^2
\]

In fact, much is known about the case when $n = 2$; it is known that all Pythagorean triples are of the form:

\begin{theorem}[Pythagorean triples]
    All pythagorean triples are of the form:

    \[
    a = r \cdot (s^2 - t^2), \qquad b = r \cdot (2st), \qquad c = r \cdot (s^2 + t^2)
    \]

    where $s > t > 0$.
\end{theorem}

The natural question to ask from such an extremely satisfying theorem is whether the same can be said for when $n \ge 2$. Initially, mathematicians set out to
to find solutions $n = 3$. However, it seemed only the "trivial" triple satisfied Fermat's equation for when $n = 2$

\[
0^3 + 1^3 = 1^3
\]

Among these mathematicians was Pierre de Fermat, who suspected it was not possible to find a nontrivial triple for the exponent $n= 3$ and what is more he believed
it was not possible to find any nontrivial triple for any exponent $n > 2$. In fact, Pierre de Fermat wrote in the margin of his copy of \textit{Arithmetic} written by Diophantus:
"
It is impossible... for any number which is a power greater than the second to be written as the sum of two like powers 
\[ 
x^n + y^n = z^n \text{ for } n > 2.
\]
I have a truly marvelous demonstration of this proposition which this margin is too narrow to contain.
"

This copy and many of Pierre de Fermat's belongings were searched in the hope of finding such a proof. Nonetheless, to this date no proof has been found.


It took Euler to provide a (flawed) proof for the nonexistence of nontrivial solutions to Fermat's equation for the exponent $n = 3$, so far so good, Fermat's conjecture held true for $n = 3$.
The case where $n = 4$ was also proved by Euler; soon enough particular cases where $n$ was some fixed natural number where being shown, which indeed seemed to suggest Fermat's conjecture was true.
However, no approach seemed to generalise to prove the general case...

The proof of Fermat's Last Theorem is the culmination of the effort of mathematicians spanning generations, and it took the direct and indirect efforts of mathematicians like: 
Gauss, Galois, Euler, Abel, Dedekind, Noether, Euler, Kummer, Mazur, Kronecker and so on to put Andrew Wiles and his collaborator Richard Taylor in the position 
to finally put to rest Fermat's Last Theorem.


\section{Formalising Fermat's Last Theorem}

Following the sequence of Lean success stories ranging from the Liquid Tensor Experiment to the formalisation of the Polynomial Freiman-Rusza conjecture. 

Prof. Kevin Buzzard from Imperial College London has received a five-year grant that will allow him to lead the formalisation of Fermat's Last Theorem. This grant kicked in in October of 2024. 

At the time of writing, since October of 2024, a digital blueprint has been set up to manage the project. Alongside other infrastructure like the project dashboard, mathematicians around the world can 
claim tasks that are set by Prof. Kevin Buzzard and if in return a task is returned with a \texttt{sorry} free proof then one can claim the glory of having completed the task.

\subsubsection{The first target of the formalisation of Fermat's Last Theorem}

The goal of the ongoing efforts of the formalisation is to reduce the proof of Fermat's Last Theorem to results that were known in the 1980s such as \text{Mazur's Theorem} 
which bounds the torsion subgroup points of an elliptic curve.
%https://imperialcollegelondon.github.io/FLT/blueprint/ch_frey.html

However, it should be mentioned that the proof being formalised is not the original proof Andrew Wiles and Richard Taylor provided in 1994, 
but a more modernised approach that has been refined over the last 20 years.

At the time of writing, the first target set by Prof. Kevin Buzzard is to formalise the \textbf{Modularity Lifting Theorem}
% https://imperialcollegelondon.github.io/FLT/blueprint/ch_overview.html#a0000000021
% , which states:

% \begin{theorem}

% \end{theorem}

After all, the ultimate goal is to formalise all of mathematics and so far the library relevant to Algebraic Number Theory, Algebraic Geometry and Arithmetic Geometry is not developed enough
to be even able to state the propositions and let alone formalise their corresponding proofs.

Morally, the goal of the formalisation of Fermat's Last Theorem is to formalise much of Algebraic Number Theory, Algebraic Geometry, Arithmetic Geometry and so forth so that one day
the mathematics library of Lean \texttt{mathlib}, contains all mathematics known to human kind.



\section{Classification of finite subgroups of the $\PGL_2(\Fbar_p)$ within Fermat's Last Theorem}

The primary goal for this project is to formalise Theorem 2.47 of \cite{dtt} which 
concerns the projective general and special linear groups:

\begin{definition}[Projective general linear group]
    The projective general linear group is the quotient group
    \[    
    \PGL_n(F) = \GL_n(F) / (Z(\GL_n(F))) = \GL_n(F) / (F^\times I) 
    \]
\end{definition}

Similarly, the Projective Special Linear Group is defined to be:

\begin{definition}[Projective special linear group]
    \[
    \PSL_n(F) = \SL_n(F) / (Z(\SL_n(F))) = \SL_n(F) / (\langle -I\rangle)
    \]
\end{definition}

The theorem states:

\begin{enumerate}
    \item If $H$ is finite subgroup of $\PGL_2(\C)$ then $H$ is isomorphic to one of the following groups: the cyclic group $C_n$ of order $n$ ($n \in \Z_{>0}$), the dihedral group $D_{2n}$ of order $2n$ ($n \in \Z_{>1}$), $A_4$, $S_4$ or $A_5$.
\item If $H$ is a finite subgroup of $\PGL_2(\Fbar_\ell)$ then one of the following holds:
\begin{enumerate}
    \item $H$ is conjugate to a subgroup of the upper triangular matrices;
    \item $H$ is conjugate to $\PGL_2(\F_{\ell^r})$ and $\PSL_2(\F_{\ell^{r}})$ for some $r \in \Z_{>0}$;
    \item $H$ is isomorphic to $A_4$, $S_4$, $A_5$ or the dihedral group $D_{2r}$ of order $2r$ for some $r \in \Z_{>1}$ not divisible by $\ell$

\end{enumerate}
    Where $\ell$ is assumed to be an odd prime.
\end{enumerate}

% Theorem 2.47 fits into 


% At first glance, neither the statement or the definitions seem to indicate how the classification of finite subgroups of $\PGL_2(\Fbar_p)$ play a role in the proof of Fermat's Last Theorem, after all, Fermat's Last Theorem is a statement
% regarding natural numbers. 

Upon inspection of the proof, it turns out that Theorem 2.47 of \cite{dtt} is required to prove the results: Theorem 2.49, Remark 2.47 and Lemma 4.11. Where Theorem 2.49 states:

\begin{theorem}[Theorem 2.49]
    Suppose $L = \Q(\sqrt{(-1)^{\ell -1}/2} \ell)$ then $\bar{\rho}$ is absolutely irreducible. Then
    there exists a non-negative integer $r$ such that for any $n \in \Z_{>0}$ we can find a
    finite set of primes $Q_n$ with the following properties.
    \begin{enumerate}
        \item If $q \in Q_n$ then $q \equiv 1 \mod n$.
        \item If $q \in Q_n$ then $\bar{\rho}$ is unramified at $q$ and $\rho(\textrm{Frob}q)$ has distinct eigenvalues.
        \item $\# Q_n = r$.
    \end{enumerate}
\end{theorem}

The place where the theorem 2.47 is of interest, the theorem that this project aims to be a blueprint for, is because proving proving the claim above requires showing that the 
cohomology group  $H^1(\textrm{Gal}(F_n / F_0), \textrm{ad}^0\bar{\rho}(1))^G_{\mathbb{Q}}$ is trivial, which in turn reduces to showing that $\ell$, an odd prime, does not divide the Galois group $\Gal(F_0 /\Q)$ which is isomorphic to a finite subgroup $\PGL_2(\Fbar_\ell)$
and has $\Gal(\Q(\zeta_\ell)/\Q)$ as a quotient.

Provided the classification of finite subgroups of $\PGL_2(\Fbar_\ell)$, it suffices to prove that the cohomology group is trivial for the case where $\ell = 3$.

% In turn, Theorem 2.49 is a key component in Theorem 3.42 which states:

% \begin{theorem}[Theorem 3.42]
%     For all finite sets $\Sigma \subset \Sigma_{\bar{\rho}}$, the map $\phi_\Sigma : R_\Sigma \rightarrow \mathbb{T}_\Sigma$ is an isomorphism and these rings are complete intersections,
% \end{theorem}

% This theorem 

% \begin{enumerate}
%     \item The local ring $R_\Sigma$ which is called the universal deformation ring for representations of type $\Sigma$.
    
%     \item The ring $\mathbb{T}_\Sigma$ is a Hecke algebra, defined as a subalgebra of the linear endomorphisms of a certain space of automorphic forms.
% \end{enumerate}

% Where $\Sigma_{\bar{\rho}}$ is the set of primes $p$ satisfying
%     \begin{itemize}
%         \item $p = \ell$ and $\bar{\rho}|_{G_{\ell}}$ is good and ordinary; or
%         \item $p \ne \ell$ and $\bar{\rho}$ is unramified at $p$.
%     \end{itemize}

% TODO - explain why this isomorphism is crucial.

% Moreover, the statement of Theorem 2.49 is the following:




% Since we have that for an odd prime $\ell > 3$ we have that $\ell$ does not divide the orders of the finite subgroups of $\PGL_2(\Fbar_\ell)$ as the finite subgroups can only have the order of the finit subgroups they are isomorphic to which
% are: $|A_4| = 4! / 2$, $|S_5| = 5 !$, $|A_5| = 5! / 2 $,  $|\PSL_2(k)| = \ell^k(\ell^{2k} - 1)$ or $|\PGL_2(k)| = (\ell^2 -1)(q^2 - q)$.



%     where furthermore, $\bar{\rho} : G_\Q \rightarrow \GL_2(k)$ is a continuous representation with the following properties
%     \begin{enumerate}
%         \item $\bar{\rho}$ is irreducible,
%         \item $\bar{\rho}$ is modular,
%         \item $\det \bar{\rho} = \epsilon$,
%         \item $\bar{\rho}\_G_{\ell}$ is semi-stable,
%         \item and if $p \ne \ell$ then $\#\bar{\rho}(I_p) | \ell$.
%     \end{enumerate} 

%     Additionally, $\mathbb{T}_\Sigma$ is the 


% However, the projective general linear group can be viewed under a different light when it is considered alongside projective space $\mathbb{P}^n = \mathbb{A}^n / \sim$ where
% these objects.

% \begin{definition}[Projective space]

% \end{definition}

% \begin{definition}[Affine space]

% \end{definition}


\section{Overview and reduction to the classification problem}

Returning to the domain of the problem of interest, classifying finite subgroups of $\PGL_2(\Fbar_p)$; one observe that $\Fbar_p$ is by construction an algebraically closed field, since it is the algebraic closure of $\F_p$,
and it turns out that for any $n \in \N$, it is possible to show that $\PGL_n(F)$ is isomorphic to $\PSL_n(F)$ for any algebraically closed field $F$, thus reducing our original classification of finite subgroups to the classification of finite subgroups of $\PSL_2(\Fbar)$.

Furthermore, on the back of the isomorphism defined between $\PGL_2(\Fbar_p)$ and $\PSL_2(\Fbar_p)$, and determining that the center of $\SL_2(\Fbar_p)$ is $Z(\SL_2(\Fbar_p)) = \langle -I\rangle$, we can in fact focus on the much more tractable problem of 
classifying the finite subgroups of $SL_2(\Fbar_p)$ to eventually classify the finite subgroups of $\PSL_2(\Fbar_p)$, and overall obtain our desired result. 

Moreover, since the more general problem of classifying the finite subgroups of $\SL_2(F)$ where $F$ is an arbitrary algebraically closed field
yields a statement that is very close to the desired statement and Christopher Butler has a in-depth exposition of this result, the formalisation of the slightly more general result was chosen.

Considering proving the existence of such an isomorphism $\PGL_2(\Fbar_p)$ and $\PSL_2(\Fbar_p)$ is no more difficult in the general case, 
the goal of the next chapter will be to formalise the definition of a suitable homomorphism between $\PGL_n(F)$ and $\PSL_n(F)$, where $F$ is an algebraically closed field, 
and formally prove in the Lean proof assistant that this homomorphism actually defines an isomorphism.

But before we dive into proving and formalising this result, we recall some preliminary results that will be used throughout the blueprint.