\chapter{Introduction}\label{Ch2_Introduction}

\section{What is the formalisation of mathematics?}

formalisation of mathematics is the art of teaching a computer what a piece of mathematics means.

That is, it is the process of carefully writing down a mathematical statement typically in first order logic or higher order logic and then scrutinously justifying each step of the proof to a computer program that checks the validity of every step of the reasoning. 

Typically one formalizes mathematics with the help of a proof assistant or interactive theorem prover, a piece of software which enables a human to write down mathematics and have the software verify the claims.

There exist many proof assistants, such examples are Lean, Isabelle, Coq, Metamath, etc.

For this project I have opted to use Lean due to its rapid growing mathematics library and its dependent type theory. I shall explain in more detail these last two reasons, but first I will comment on what Lean is.

\subsubsection{What is Lean?}

Lean is both a functional programming language and an interactive theorem prover that is being developed at Microsoft research and AWS by Leonardo de Moura and his team.
It has been designed for both use in cutting-edge mathematics and the verification of software which is often essential to safety critical systems where correctness is of extreme
TODO:


- Brief explanation of type theory and curry-howard isomorphism.

% Proof that the sum of odd numbers are the squares.
- Example of formal proof and comparison with informal proof.


\begin{verbatim}
theorem add\textunderscore comm (a b : Nat) : a + b = b + a :=
  Nat.add\textunderscore comm a b
\end{verbatim}


\section{Fermat's Last Theorem}


\subsubsection{Problem statement and its history}
Fermat's Last Theorem, before it was proved that is, A conjecture about the \textit{Fermat equation} which is defined to be

\begin{definition}[Fermat Equation]
    The equation $a^n + b^n = c^n$ is Fermat's Equation
\end{definition}

When $a, b, c$ and $n$ in this equation are restricted to positive integers, we are defining a particular family of what are called \textit{Diophantine equation}.
Diophantus, an ancient greek mathematician was interested in positive integers which satisfy this equation. For instance, a particular set of numbers which satisfy this equation 
are the \textit{Pythagorean triples}, such triples have been known since Babylonian times. For example, when we substitute the Pythagorean triple $(a,b,c) = (3,4,5)$ and set $n = 2$ we find that 
indeed Fermat's equation holds for this choice of numbers since:

\[
3^2 + 4^2 = 5^2
\]

In fact, much is known about the case when $n = 2$; it is known that all Pythagorean triples are of the form:

\begin{theorem}[Pythagorean triples]
    All pythagorean triples are of the form:

    \[
    a = r \cdot (s^2 - t^2), \qquad b = r \cdot (2st) = r \cdot (s^2 + t^2)
    \]
\end{theorem}

The natural question to ask from such an extremely satisfying theorem is whether the same can be said for when $n \ge 2$. Initially, mathematicians set out to
to find solutions $n = 3$. However, it seemed only the "trivial" triple satsfied Fermat's equation for when $n = 2$

\[
0^3 + 1^3 = 1^3
\]

Among these mathematicians was Pierre de Fermat, who suspected it was not possible to find a nontrivial triple for the exponent $n= 3$ and what is more he believed
it was not possible to find any nontrivial triple for any exponent $n > 2$. In fact, Pierre de Fermat wrote in the margin of his copy of \textit{Arithmetic} written by Diophantus:
"
It is impossible... for any number which is a power greater than the second to be written as the sum of two like powers 
\[ 
x^n + y^n = z^n \text{ for } n > 2.
\]
I have a truly marvelous demonstration of this proposition which this margin is too narrow to contain.
"

This copy and many of Pierre de Fermat's belongings were searched in the hope of finding such a proof. Nonetheless, to this date no proof has been found.


It took Euler to provide a (flawed) proof for the nonexistence of nontrivial solutions to Fermat's equation for the exponent $n = 3$, so far so good, Fermat's conjecture held true for $n = 3$.
The case where $n = 4$ was also proved by Euler. %Fermat's conjecture was shown to be true for particular values of $n > 2$.  

TODO

\subsubsection{History of Fermat's Equation}

The proof of Fermat's Last Theorem is the culmination of countless contributions which span many generations.

From Diophantus, the first known person to systematically study what we now call \textit{Diophantine equations}, to Fermat developing the elementary theory of number theory and then due to the invaluable work of countless mathematicians 
around the world which built upon each other's work a list of such mathematicians contains the names of: Gauss, Galois, Euler, Abel, Dedekind, Noether, Euler, Kummer, Mazur, Kronecker, etc.

TODO

\section{Formalizing Fermat's Last Theorem}

Following the sequence of success stories ranging from the Liquid Tensor Experiment to the formalisation of the Polynomial Freiman-Rusza conjecture. 

Prof. Kevin Buzzard from Imperial College London has received a five-year grant that will allow him to lead the formalisation of Fermat's Last Theorem. This grant kicked in in October of 2024. 

At the time of writing, since October of 2024, a digital blueprint has been set up to manage the project.

Alongside other infrastructure like the project dashboard, mathematicians around the world can claim tasks that are set by Prof. Kevin Buzzard and if in return a task is returned with a "sorry" free proof then one can claim the glory of having completed the task.

\subsubsection{Goal of the formalisation of Fermat's Last Theorem}

The goal of the ongoing efforts of the formalisation is to reduce the proof of Fermat's Last Theorem to results that were known in the 1980s such as \text{Mazur's Theorem}.

However, it should be mentioned that the proof being formalised is not the proof Andrew Wiles and Richard Taylor initially came up with during 1994, but a more modernised approach that has been refined over the last 20 years.

At the time of writing, the first target set by Prof. Kevin Buzzard is to formalise the \textbf{Modularity Lifting Theorem}
% https://imperialcollegelondon.github.io/FLT/blueprint/ch_overview.html#a0000000021
% , which states:

% \begin{theorem}

% \end{theorem}

After all, the ultimate goal is to formalise all of mathematics and so far the library relevant to Algebraic Number Theory, Algebraic Geometry and Arithmetic Geometry is not developed enough
to be even able to state the propositions and let alone formalise their corresponding proofs.

Morally, the goal of the formalisation of Fermat's Last Theorem is to formalise much of Algebraic Number Theory, Algebraic Geometry, Arithmetic Geometry and so forth so that one day
the mathematics library of Lean contains all mathematics known to human kind.



\subsubsection{Classification of finite subgroups of $\PGL_2(\Fbar_p)$}





\section{Classification of finite subgroups of the $\PGL_2(\Fbar_p)$}

TODO:

-Why are the finite subgroups of  $\PGL_2(\bar{\F}_p)$ relevant to number theory: i.e: Automorphic forms, Galois representations, etc.


The primary concern of this project is to formalise Theorem 2.47 of [DTT] which states:

\begin{enumerate}
    \item If $H$ is finite subgroup of $\PGL_2(\C)$ then $H$ is isomorphic to one of the following groups: the cyclic group $C_n$ of order $n$ ($n \in \Z_{>0}$), the dihedral group $D_{2n}$ of order $2n$ ($n \in \Z_{>1}$), $A_4$, $S_4$ or $A_5$.
\item If $H$ is a finite subgroup of $\PGL_2(\Fbar_p)$ then one of the following holds:
\begin{enumerate}
    \item $H$ is conjugate to a subgroup of the upper triangular matrices;
    \item $H$ is conjugate to $\PGL_2(\F_{\ell^r})$ and $\PSL_2(\F_{\ell^{r}})$ for some $r \in \Z_{>0}$;
    \item $H$ is isomorphic to $A_4$, $S_4$, $A_5$ or the dihedral group $D_{2r}$ of order $2r$ for some $r \in \Z_{>1}$ not divisible by $\ell$

\end{enumerate}
    Where $\ell$ is assumed to be an odd prime.
\end{enumerate}


The Projective General Linear Group is:

\begin{equation}
    \PGL_n(F) = \GL_n(F) / (Z(\GL_n(F))) = \GL_n(F) / (F^\times I) 
\end{equation}

Similarly, the Projective Special Linear Group is:

\begin{equation}
    \PSL_n(F) = \SL_n(F) / (Z(\SL_n(F))) = \SL_n(F) / (\langle -I\rangle)
\end{equation}

Given we are working over an algebraically closed field $F$, it turns out that for any $n \in \N$, $\PGL_n(F)$ is isomorphic to $\PSL_n(F)$.

This isomorphism will be crucial as it will allow us to focus on classifying finite subgroups of $SL_2(F)$ to classify the finite subgroups of $\PGL_2(F)$.

The goal of the next chapter is to prove and formalize this result.